\documentclass[12pt,a4paper]{article}
\usepackage[top=2cm,bottom=2cm,left=2.5cm,right=2.5cm]{geometry}
\usepackage[latin1]{inputenc}
\usepackage[brazilian]{babel}
\usepackage[document]{ragged2e}
\usepackage{graphicx}
\usepackage[table]{xcolor}
\usepackage{floatrow}
\DeclareFloatFont{tiny}{\tiny}
\floatsetup[table]{font=tiny}
\setlength{\arrayrulewidth}{1mm}
\setlength{\tabcolsep}{18pt}
\renewcommand{\arraystretch}{2.5}

\begin{document}
\begin{titlepage}
\title{ \begin{huge} \textbf{Pesquisa Sobre o uso de Operadores Bin�rios em Linguagem C} \end{huge}\\
\large Para In�cio das Tarefas de Inicia��o Cient�fica 2017/01 \\ Engenharia de Controle e Automa��o }
\author{Autores: Mateus Braga e Matheus Cardoso \\ Orientador: M�rcio Assun��o \\ Professor: L�cio Passos}
\maketitle
\end{titlepage}
\newpage
%--------------------------------------------------------------------------------Op----------------------
\section{Tabela de Operadores Bin�rios}
\begin{center}
{\rowcolors{2}{green!80!yellow!50}{green!70!yellow!40}
\begin{tabular}{ |p{2.5cm}|p{2cm}| }
\hline
\multicolumn{2}{|c|}{Device Features} \\
\hline
Operador & Nome\\
\hline
\& & AND \\
$|$ & OR inclusivo\\
. \textbf{\^}  & OR exclusivo\\
$<<$ & Deslocamento � esquerda\\
$>>$ & Deslocamento � direita\\
. \textbf{\~} & Complemento\\
\hline
\end{tabular}
}
\end{center}
%--------------------------------------------------------------------------------------------------------
\section{Operador}
\subsection{And \&}
\justify
\quad
O operador And funciona comparando bit-a-bit de um n�mero com outro, de forma que funciona semelhante a uma porta l�gica And.\\[-0.7cm]
\justify
\quad
Porta AND:\\
\includegraphics[width=30mm,scale=0.30]{portlog}\\
Exemplo em C�digo:\\
\includegraphics[width=150mm,scale=1.50]{C1}\\

\subsection{OR inclusivo \|}
\justify
\quad
O operador OR inclusivo funciona comparando bit-a-bit de um n�mero com outro, de forma que se um dos bits comparados for 1 a sa�da j� � automaticamente 1. Funciona como uma porta OR\\[-0.7cm]
\justify
\quad
Porta OR:\\
\includegraphics[width=30mm,scale=0.30]{portlog2}\\
Exemplo em C�digo:\\
\includegraphics[width=150mm,scale=1.50]{C2}\\

\subsection{OR exclusivo \|}
\justify
\quad
O operador OR exclusivo funciona comparando bit-a-bit de um n�mero com outro, de forma que se os bits comparados forem diferentes a sa�da ser� 1. Funciona como a porta XOR.\\[-0.7cm]
\justify
\quad
Porta XOR:\\
\includegraphics[width=80mm,scale=0.80]{portlog3}\\
Exemplo em C�digo:\\
\includegraphics[width=150mm,scale=1.50]{C3}\\

\subsection{Deslocamento a Esquerda $<<$}
\justify
\quad
O operador Deslocamento a esquerda funciona deslocando todos os bits do n�mero para esquerda e completando a �ltima casa com zero. Funciona como uma multiplica��o por 2.\\[-0.7cm]
\justify
\quad
Exemplo em C�digo:\\
\includegraphics[width=180mm,scale=1.80]{C4}\\

\subsection{Deslocamento a Direita $>>$}
\justify
\quad
O operador Deslocamento a direita funciona deslocando todos os bits do n�mero para direita e completando a primeira casa com zero. Funciona como uma divis�o por 2.\\[-0.7cm]
\justify
\quad
Exemplo em C�digo:\\
\includegraphics[width=180mm,scale=1.80]{C5}\\

\subsection{Complemento}
\justify
\quad
O operador complemento funciona realizando a t�cnica de inverter bit-a-bit de um n�mero.Dessa forma, assim como o operador de deslocamento a direita e deslocamento a esquerda, atua sobre um �nico n�mero. O seu resultado sempre ser� o n�mero negativo com seu m�dulo acrescido de uma unidade (estar� escrito na forma de complemento de 2). Aplicar essa opera��o duas vezes em um n�mero sempre resultado nele mesmo.\\[-0.7cm]
\justify
\quad
Exemplo em C�digo:\\
\includegraphics[width=180mm,scale=1.80]{C6}\\
\pagebreak

\begin{thebibliography}{}
\bibitem {embarcados} https://www.embarcados.com.br/xor/\\
\bibitem {dpi} http://www.dpi.inpe.br/\~carlos/Academicos/Cursos/ArqComp/aula\_5bn1.html\\
\bibitem {pythontutor} http://www.pythontutor.com/c.html\#mode=edit \\
\bibitem {wikipedia} https://en.wikipedia.org/wiki/Bitwise\_operations\_in\_C \\
\bibitem {tutorialspoint} http://www.tutorialspoint.com/cprogramming/c\_bitwise\_operators.htm \\

\end{thebibliography}


\end{document}